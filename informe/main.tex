\documentclass{article}

% ----------------------- PAQUETES ---------------------- %
% Set the font (output) encodings
\usepackage[T1]{fontenc}
\usepackage[spanish]{babel}
\usepackage{graphicx}							% paquete para poder usra graficos
\usepackage{float}								% paquete para poder usar \begin {figure}[H]
\usepackage[hyphens]{url}					% https://stackoverflow.com/questions/4146606/wrap-url-ignores-margin-in-bibtex-using-pdflatex
\usepackage{hyperref}							% paquete para hacer hyperreferencias a links
\usepackage{fancyhdr}							% paquete para poner cosas en footer y header
\usepackage{wrapfig}							% paquete para hacer wrapfigure
\usepackage{titling}							% paquete para modificar el espaciado de los titulos
\usepackage{titlesec}							% paquete para modificar como se ven los titulos
\usepackage{minted}								% paquete para agregar codigo al documento
\usepackage[margin=1in]{geometry} % paquete para meter un margen de una pulgada
\usepackage{todonotes}



% ------------------------------------------------------- %

% ---------------------- CONFIGURACION ------------------ %

\setlength{\headheight}{12pt} %esto era para evitar un error de hbox
\graphicspath{{images/}}
% ----------------- Configuracion de hyperref ----------- %
\hypersetup{								
	colorlinks=true,
	linkcolor=black,			%modo claro
	%linkcolor=white,		%modo oscuro
	filecolor=brown,		
	urlcolor=blue,
	pdftitle={Proyecto Indoor},
	}
% ------------------------------------------------------- %

% ----------------- Configuracion de fancyhdr --------------- %
\pagestyle{fancy}
\setlength{\headheight}{12.17pt} %esto esta para solucionar un warning del fancyhdr
\rfoot{Página \thepage}
\cfoot{}
\lfoot{Krapp Ramiro}
\renewcommand{\sectionmark}[1]{%
\markboth{\thesection\quad #1}{}}
\fancyhead[L]{\leftmark}
\fancyhead[R]{Traintorio}
\renewcommand{\headrulewidth}{2pt}
\renewcommand{\footrulewidth}{1pt}
% ------------------------------------------------------- %


%-------------- Formatos de los titulos --------------%
\newcommand{\sectionbreak}{\clearpage}

\titleformat{\section}
	{\bfseries \huge}
	{}
	{0em}
	{}[\titlerule]

\titleformat{\subsection}
	{\bfseries \Large}
	{}
	{0em}
	{}

\titleformat{\subsubsection}
	{\bfseries \large}
	{}
	{0em}
	{}

\titlespacing{\section} %me permite controlar el espaciado de la seccion que le indico
	{0em}
	{0em}
	{1.5em}

\titlespacing{\subsection}
{0em} %sangria
{3em} %separacion con lo que hay arriba
{0.5em} %separacion con lo que hay abajo
%----------------------------------------------------- %

% ----------------- FIN DE CONFIGURACION ------------- %


% Spanish-specific commands
\begin{document}
\begin{titlepage}
	\begin{center}
		\vspace{1cm}

		{\Huge
			\textbf{Traintorio}}

		\vspace{0.3cm}
		{\LARGE
			Informe técnico}

		\vspace{0.5cm}
		{\Large
			Un trabajo presentado para la materia de \\
			Proyectos y Diseño Electrónico}

		\vspace{2cm}

		\begin{figure}[H]
			\centering
			\includegraphics[width=0.6\textwidth]{logo.png}
		\end{figure}

		\vfill

		{\Large
			\textbf{Krapp Ramiro} \\
			\vspace{0.5cm}
			Instituto tecnológico San Bonifacio\\
			Departamento de electrónica\\
			\today
		}

		\vspace{0.5cm}
		{\large Hecho en {\LaTeX}\\
			Versión Alpha 0.1}

	\end{center}
\end{titlepage}

\tableofcontents		\noindent\rule{\textwidth}{0.7pt}
El índice tiene hipervínculos incorporados!
Toca en cada seccion y automaticamente tu lector de pdfs te llevara a esa página
\\[12pt] %esto esta para generar un espaciado entre lineas
{\large Tengo un
\href{https://github.com/KrappRamiro/traintorio}{Repositorio en GitHub}}\\
{\small \url{https://github.com/KrappRamiro/traintorio}}



\section{Introducción}

\section{Diagrama esquematico}
%\includegraphics[width=0.9\textwidth]{../../diagrama_drawio.png}

\section{Base de datos}
%\includegraphics[width=0.9\textwidth]{../../database_schema.png}

\section{Codigo del programa}
\inputminted %TODO: ponerle inconsolata
[frame= lines, linenos, breaklines, tabsize = 3, fontsize=\footnotesize,
	label=Codigo principal]
{cpp}{../src/main.cpp}

\section{Bitacoras Personales}
\subsection{Krapp Ramiro}
\subsubsection{24/03/2022}
\begin{itemize}
	\item Comence creando un repositorio en github para subir todos los cambios del proyecto
	\item Cree un codigo en C++, para definir un sistema de clases.
	      La idea es hacer una clase Tren, para que sirva de blueprint para todos los trenes,
	      y una clase Persona, para que sea padre de otras dos clases, Maquinista y Pasajero.
	      Al pasajero le voy a asignar una sube, y al maquinista le voy a asignar
	      un salario y un seniority
\end{itemize}
\subsubsection{25/03/2022}
\todo{Hacer las urls mas chicas con \small o tiny}
\begin{itemize}
	\item Pienso implementar la sube con un sistema usando RFID\\
	      \small{\url{https://randomnerdtutorials.com/security-access-using-mfrc522-rfid-reader-with-arduino/}}
	\item La idea seria armar un sistema en el que cada usuario pueda tener un llavero
	      RFID, y que asigne ese llavero RFID con una cuenta.
	      Tambien necesito comprar los lectores para RFID.
	      En total, tengo pensado comprar 2 lectores y 4 llaveros RFID.
	      Por qué 2 lectores? Estaba pensando en asignar cada uno a una estación distinta.
	      Por qué 4 llaveros? Estaba pensando en asignar cada uno a un pasajero distinto.
	\item Encontre que para en \LaTeX \ dejar de tener problema con las url yendose fuera
	      pantalla, puedo usar el paquete url con la opcion [hyphens], lo
	      unico es que hay que cargar este paquete antes de hyperref.
	      Esto es porque por defecto el paquete hyperref ya carga al paquete url
	      \url{https://tex.stackexchange.com/questions/544671/option-clash-for-package-url-urlstyle}

\end{itemize}
\subsubsection{26/03/2022}
\begin{itemize}
	\item Encontre mucha documentacion del ESP32 y de proyectos con el RFID, la principal es esta:
	\item \url{https://arduinogetstarted.com/tutorials/arduino-rfid-nfc}
	\item \url{https://olddocs.zerynth.com/latest/official/board.zerynth.doit_esp32/docs/index.html}
	\item \url{https://testzdoc.zerynth.com/reference/boards/doit_esp32/docs/}
	\item \url{https://randomnerdtutorials.com/esp32-pinout-reference-gpios/}
	\item \url{https://randomnerdtutorials.com/getting-started-with-esp32/}
	\item Voy a usar el grafico de randomnerdtutorials, del link de getting-started...,
	      el que incluye que pines son GPIO, me va a servir un montón.
	      Para cuando quiera programar, solamente tengo que recordar que lo mejor es usar los
	      GPIO del 13 al 33, y que mi DOIT ESP32 DevKit V1 es la version de 30 pines
	\item Decidi seguir el tutorial de este link \url{https://www.instructables.com/ESP32-With-RFID-Access-Control/}

\end{itemize}


\end{document}