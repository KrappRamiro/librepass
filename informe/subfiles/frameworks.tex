\documentclass[../informe_krapp.tex]{subfiles}
\begin{document}
\renewcommand{\subsectionbreak}{}
\section{Frameworks}
Un framework es un esquema o marco de trabajo que ofrece una estructura base para elaborar un proyecto con objetivos específicos, una especie de plantilla que sirve como punto de partida para la organización y desarrollo de software. Utilizar frameworks puede simplificar (y mucho) una tarea o proceso.

Generalmente, los frameworks son usados por programadores porque permiten acelerar el trabajo y favorecer que este sea colaborativo, reducir errores y obtener un resultado de más calidad

Un framework sirve para acometer un proyecto en menos tiempo, y en el sector de la programación, con un código más limpio y consistente, de manera rápida y eficaz. El framework ofrece una estructura base que los programadores pueden complementar o modificar según sus objetivos.
% -------------- ARDUINO -------------------%
\subsection{Arduino}
\begin{wrapfigure}{r}{0in}
	\includegraphics[width=2in, keepaspectratio]{arduino.png}
\end{wrapfigure}
El framework Arduino, que provee una amplia variedad de
clases, metodos y funciones útiles para el desarrollo en sistemas embebidos.
Este framework se implementó a través de platformIO, una extensión de Visual Studio Code.
En este proyecto, todo el desarrollo del ESP32 fue hecho mediante este framework.

\subsubsection{Razones por las que usé Arduino}
Use Arduino porque es un framework con el que ya estoy familiarizado, además que, gracias a la existencia de librerias como Wifi.h, HTTPClient.h, MFRC522, Wire.h, etc..., gran cantidad del desarrollo ya está hecho, y lo único que tengo que hacer es construir alrededor de esas librerías.

Además, el uso de Arduino se está volviendo un estandar en la industria, ya que permite que todos los programadores sigan una misma forma de desarrollo,
cosa que resultaba complicada en C++, el cual es un lenguaje conocido por permitir multiples formas de desarrollar una misma cosa, o dicho por el propio creador, ``multiples formas de pegarse un tiro en el pie''
\clearpage

% -------------- FLASK -------------------%
\subsection{Flask}
\begin{figure}[H]
	\includegraphics[width=0.7\textwidth, keepaspectratio]{flask.png}
	\centering
\end{figure}
Flask es un microwebframework usado para, como indica el nombre, crear aplicaciones webs. Este, al ser modular y escalable, tiene la posibilidad
de permitir el agregado de ORM's (Object Relational Manager), routers, renderizador de templates, sistema de logins,
forms, etc...
En este proyecto, Flask se usó para el desarrollo del backend, y con la ayuda del renderizador de templates Jinja2 y Nunjucks, se templatizo el frontend.

\subsubsection{Razones por las que usé Flask}
Use Flask porque me lo aconsejó un programador conocido, ya que aprender a usarlo me iba a abrir las puertas al desarrollo backend con otros
frameworks como Django, los cuales tienen un gran uso en el mercado laboral.
Ademas, Flask me permite desarrollar mi programa de forma sencilla gracias a su alta modularidad


% -------------- BOOTSTRAP -------------------%
\subsection{Bootstrap}
\begin{wrapfigure}{r}{0in}
	\includegraphics[width=2in, keepaspectratio]{bootstrap.png}
\end{wrapfigure}
Bootstrap es el framework CSS más popular para desarrollar aplicaciones responsivas y aptas para dispositivos móviles.
En este proyecto, se usó la versión 5 de Bootstrap para el desarrollo del frontend.

Este framework cuenta con multiples clases, las cuales se usan para crear las páginas webs al gusto del programador.

\subsubsection{Razones por las que usé Bootstrap}
La principal razón por la que use bootstrap es que ya tiene componentes creados previamente, lo cual libera carga al desarrollador, y permite en concentrarse en partes más esenciales del proyecto.
\clearpage


% -------------- AWS -------------------%
\subsection{Amazon Web Services (AWS)}
\begin{wrapfigure}{r}{0in}
	\includegraphics[width=2in, keepaspectratio]{aws.jpg}
\end{wrapfigure}
Amazon Web Services (AWS) es la plataforma cloud más adoptada en el mundo, con más de 200 servicios distintos.
En este proyecto, AWS se usó para la implementación de la infrastructura de la webapp.
Se usarón los servicios de Relational Database Service (RDS) y Elastic Beanstalk.

\subsubsection{Razones por las que usé AWS}
Hay dos razones principales por las que usé AWS: La primera razón es porque hostear un servidor físico requiere mucho tiempo y esfuerzo, además de que es \textbf{muy caro}, mientras que AWS es completamente escalable, y en tamaños chicos es realmente económico. La segunda es que en la industria ya no se está acostumbrando tanto a usar servidores físicos, sino que se están realizando muchas migraciones a servicios en la nube gracias a la comodidas que proveen.

Y en el ámbito de la comodidad, su servicio ElasticBeanstalk facilita mucho el desarrollo de aplicaciones web, ya que en ningun momento tuve que configurar más que el PYTHONPATH y el WSGIPATH, y es simplemente un archivo de texto, además de acoplarle la base de datos postgreSQL

% -------------- PostgreSQL -------------------%
\subsection{PostgreSQL}
\begin{wrapfigure}{r}{0in}
	\includegraphics[width=2in, keepaspectratio]{postgreSQL.png}
\end{wrapfigure}
PostgreSQL (tambien conocido como Postgres por la comunidad) es un sistema de gestión de bases de datos relacional orientado a objetos y de código abierto.

Como muchos otros proyectos de código abierto, el desarrollo de PostgreSQL no es manejado por una empresa o persona, sino que es dirigido por una comunidad de desarrolladores que trabajan de forma desinteresada, altruista, libre o apoyados por organizaciones comerciales. Dicha comunidad es denominada el PGDG (PostgreSQL Global Development Group).

\subsubsection{Razones por las que usé PostgreSQL}
La principal razón por la que use PostgreSQL es que es un estandar en la industria moderna a la hora de hacer bases de datos SQL.

\end{document}